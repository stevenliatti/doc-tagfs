%%%%%%%%%%%%%%%%%%%%%%%%%%%%%%%%%%%%%%%%%%%%%%%%%%%%%%%%%%%%%%%%%%%%%%%%%%%%%%%%%%%%%%%%%%%%%%%%%%%
%%%%%%%%%%%%%%%%%%%%%%%%%%%%%%%%%%%%%%%%%%%%%%%%%%%%%%%%%%%%%%%%%%%%%%%%%%%%%%%%%%%%%%%%%%%%%%%%%%%
\subsection{Rust VS C}
Dans cette sous-section, nous allons discuter de l'implémentation du système en Rust plutôt qu'en 
C. Les avantages sont nombreux et les inconvénients sont acceptables.

\subsubsection{Avantages de Rust par rapport à C}
L'avantage principal de Rust par rapport à C est la garantie sur la sécurité de la mémoire (\textit{safe}). 
Tant que le code compile, il sera sûr en termes d'allocation/désallocation mémoire et concurrence 
(les erreurs de logique sont néanmoins toujours à la charge du programmeur). Cette garantie est 
fournie par le compilateur de Rust, qu'on pourrait définir d'"intelligent". Intelligent car la 
quasi-totalité des erreurs liées à la mémoire sont détectées à la compilation. Il va même plus 
loin en étant très verbeux sur les erreurs survenues, en suggérant souvent le moyen de résoudre 
le problème. Ce qui, avec le temps, sont devenues des bonnes pratiques à connaître lorsque 
nous programmons en C, Rust force les programmeur à les implémenter avant même que le code soit 
compilé. Les listings de code suivants illustrent deux exemples où Rust se comporte "mieux" que 
C. Les listings \ref{undefined_pointer_c} et \ref{undefined_pointer_rust} montrent une fonction, 
respectivement écrite en C et en Rust, qui retourne un pointeur indéfini. En C, la détection de 
l'erreur surviendrait à l'exécution du programme (bien que selon le compilateur C utilisé, un 
avertissement, \textit{warning}, sera affiché au programmeur), avec le fameux \textit{Segmentation fault}.
L'équivalent Rust refuserait de compiler ce code, à cause de la règle de la durée de vie non respectée.
\bigbreak
\begin{code}
    \begin{minted}[bgcolor=mygray,breaklines,breaksymbol=,linenos,frame=single,stepnumber=1,tabsize=2]{c}
int* undefined_pointer() {
    int data[10] = {114, 117, 115, 116, 105, 115, 98, 101, 115, 116};
    return &data[3];
}
    \end{minted}
    \caption{Création d'un pointeur indéfini en C}
    \label{undefined_pointer_c}
\end{code}
\bigbreak
\begin{code}
    \begin{minted}[bgcolor=mygray,breaklines,breaksymbol=,linenos,frame=single,stepnumber=1,tabsize=2]{rust}
fn undefined_pointer() -> &i32 {
    let data = [114, 117, 115, 116, 105, 115, 98, 101, 115, 116];
    &data[3]
}
    \end{minted}
    \caption{Création d'un pointeur indéfini en Rust}
    \label{undefined_pointer_rust}
\end{code}
\bigbreak
Un autre exemple est l'accès non autorisé à de la mémoire. Les listings \ref{memory_access_c} 
et \ref{memory_access_rust} montrent une fonction, respectivement écrite en C et en Rust, qui 
imprime sur la sortie standard un élément d'un tableau à un index non correct. En C, "aucun problème", 
la fonction imprimera ce qu'elle trouve à cette adresse mémoire, sans se plaindre. Ce genre 
d'erreurs est potentiellement une vulnérabilité qui peut être exploitée à des fins malicieuses.
En Rust, cette erreur ne serait pas détectée à la compilation, mais au moins le programme 
s'arrêterait abruptement à son exécution (\textit{panicked}) et l'erreur pourrait être corrigée.
\bigbreak
\begin{code}
    \begin{minted}[bgcolor=mygray,breaklines,breaksymbol=,linenos,frame=single,stepnumber=1,tabsize=2]{c}
void print_data() {
    int data[10] = {114, 117, 115, 116, 105, 115, 98, 101, 115, 116};
    printf("%d\n", data[10]);
}
    \end{minted}
    \caption{Accès non autorisé à de la mémoire en C}
    \label{memory_access_c}
\end{code}
\bigbreak
\bigbreak
\begin{code}
    \begin{minted}[bgcolor=mygray,breaklines,breaksymbol=,linenos,frame=single,stepnumber=1,tabsize=2]{rust}
fn print_data() {
    let data = [114, 117, 115, 116, 105, 115, 98, 101, 115, 116];
    println!("{}", data[10]);
}
    \end{minted}
    \caption{Accès non autorisé à de la mémoire en Rust}
    \label{memory_access_rust}
\end{code}
\bigbreak
Nous pouvons également citer que Rust est parfois plus rapide que C pour certaines tâches, ou 
presque aussi rapide. Le site \textit{The Computer Language Benchmarks Game} \cite{ref51} dresse 
une comparaison entre Rust et, un à un C, C++, Go et Java sur les temps d'exécution, quantité de 
mémoire et charge processeur pour une variété d'applications. Rust comparé à C et C++ est parfois 
plus rapide et souvent distancé de peu. Rust comparé à Go et Java est toujours le plus rapide. 
On peut en conclure que sécurité peut rimer au propre comme au figuré avec rapidité.
\bigbreak
Un autre grand avantage de Rust sur C est sa gestion des erreurs et l'absence de \mintinline{c}{NULL}. 
Tony Hoare, un chercheur en informatique, est l'inventeur de \mintinline{c}{NULL} et l'appelle 
aujourd'hui son erreur à un milliard de dollars \cite{ref52}. \mintinline{c}{NULL} autorise de 
nombreuses erreurs inattendues, malgré le fait que l'idée n'est pas mauvaise (il est pratique de 
déclarer une variable ne contenant pas de valeur). En Rust, \mintinline{c}{NULL} n'existe pas, à la 
place une énumération très puissante est disponible, \mintinline{rust}{Option} (voir listing \ref{rust_option}).
Elle évite ainsi de manipuler des données vides.
\bigbreak
Viennent ensuite plusieurs avantages moins déterminants, mais toutefois bien pratiques dans la 
vie de tous les jours du programmeur. Citons par exemple le formidable outil Cargo et la source 
de \textit{crates} disponible sur \href{crates.io}{crates.io}, contenant des milliers de librairies 
réalisées et éprouvées par la communauté. Les tests unitaires du programme sont intégrés directement 
au langage et leur exécution est facilitée avec Cargo. La librairie standard de Rust est bien fournie,
notamment avec les collections (vecteurs, tables de hachage, etc.). Enfin, s'il fallait terminer, 
la gestion des types génériques directement inclue au langage est un atout bien pratique.

\subsubsection{Inconvénients de Rust par rapport à C}
Rust a malgré tout deux défauts, plus ou moins handicapants selon le temps et la situation, liés 
l'un à l'autre. Le premier est la courbe d'apprentissage du langage pouvant être longue et décourageante. 
Rust est régi par certaines règles contraignantes, simples mais pas faciles à cerner dans toutes 
les situations. C'est notamment le cas des règles d'\textit{ownership}, de \textit{borrowing} et 
des références (voir sous-section \ref{rust_ownership_borrowing}), assez uniques à Rust. 
Le deuxième défaut découle du premier : il faut parfois revoir certains algorithmes ou structures 
de données facilement implémentables en C ou autres langages similaires. C'est notamment ce qu'il 
s'est produit durant l'implémentation de Tag Engine et sa structure de données pour contenir 
l'arborescence des répertoires, fichiers et tags (voir sous-section \ref{problemes} pour plus de détails).
D'autres petits défauts peuvent être mentionnés, comme par exemple le manque d'arguments par défaut 
lors des déclarations de fonctions ou la surcharge de méthodes ou fonctions. Ce ne sont pas des 
défauts à proprement parler, mais des facilités qu'il aurait été souhaitable de disposer. 
Finalement, le frein majeur à l'adoption massive du langage (comme bien d'autres nouveau langages) 
est le manque de soutien de la part d'une importante société (même si Mozilla l'utilise pour 
son navigateur Firefox) et la "flemme" des programmeurs de se détourner de C ou C++.

%%%%%%%%%%%%%%%%%%%%%%%%%%%%%%%%%%%%%%%%%%%%%%%%%%%%%%%%%%%%%%%%%%%%%%%%%%%%%%%%%%%%%%%%%%%%%%%%%%%
%%%%%%%%%%%%%%%%%%%%%%%%%%%%%%%%%%%%%%%%%%%%%%%%%%%%%%%%%%%%%%%%%%%%%%%%%%%%%%%%%%%%%%%%%%%%%%%%%%%
\subsection{Problèmes rencontrés}\label{problemes}
Le principal problème rencontré durant ce travail a été l'implémentation d'un arbre en Rust pour 
les besoins de Tag Engine. L'implémentation simple d'un noeud d'un arbre en C est similaire à celle 
décrite au listing \ref{problemes_tree_c}.
\bigbreak
\begin{code}
    \begin{minted}[bgcolor=mygray,breaklines,breaksymbol=,linenos,frame=single,stepnumber=1,tabsize=2]{c}
struct Node {
    int data; // le type n'est pas important pour l'exemple
    struct Node** children;
};
    \end{minted}
    \caption{Implémentation de la structure d'un arbre en C}
    \label{problemes_tree_c}
\end{code}
\bigbreak
C'est une structure récursive et relativement facile à comprendre : un noeud est constitué d'une 
donnée d'un type choisi et d'un tableau de pointeurs vers des noeuds enfants. Ce genre de déclaration 
est impossible en Rust, car le compilateur a besoin de connaître la taille exacte des données à la 
compilation (éviter les structure récursives infinies). Une solution est d'envelopper les noeuds 
enfants dans un pointeur intelligent de type \mintinline{rust}{Box} (voir chapitre 15 du 
\textit{book} \cite{ref0}). Mais un autre problème surviendra au moment de l'utilisation des noeuds, 
car Rust ne permet pas plusieurs références mutables vers une même valeur. Pour résoudre ce dernier 
problème, l'utilisation des \mintinline{rust}{Rc<T>}, ou \textit{reference counting}. C'est un autre 
pointeur intelligent, qui autorise plusieurs \textit{ownership} pour une seule valeur. On se 
retrouve alors avec une imbrication de pointeurs avancés, comme dans le listing \ref{node_rust_leipzig}, 
solution proposée par Rust Leipzig \cite{ref49} :
\bigbreak
\begin{code}
    \begin{minted}[bgcolor=mygray,breaklines,breaksymbol=,linenos,frame=single,stepnumber=1,tabsize=2]{rust}
use std::rc::Rc;
use std::cell::RefCell;

struct Node<T> {
    previous: Rc<RefCell<Box<Node<T>>>>,
    //        ^  ^       ^   ^
    //        |  |       |   |
    //        |  |       |   - The next Node with generic `T`
    //        |  |       |
    //        |  |       - Heap allocated memory, needed
    //        |  |         if `T` is a trait object.
    //        |  |
    //        |  - A mutable memory location with
    //        |    dynamically checked borrow rules.
    //        |    Needed because `Box` is immutable.
    //        |
    //        - Reference counted pointer, will be
    //          dropped when every reference is gone.
    //          Needed to create multiple node references.

    next: Vec<Rc<RefCell<Box<T>>>>,
    data: T,
    // ...
}
    \end{minted}
    \caption{Structure d'un noeud en Rust avec pointeurs intelligents - \cite{ref49}}
    \label{node_rust_leipzig}
\end{code}
\bigbreak
Non seulement cette solution n'est pas très lisible mais en plus elle a le défaut de supprimer 
tous les avantages qu'offre le compilateur Rust sur les contraintes de portée et durée de vie 
des variables. Une solution qui semble résoudre tous les problèmes précédents est décrite au 
listing \ref{rust_leipzig_arena}. Elle utilise une \textit{arena}, une région mémoire \cite{ref53}. Plutôt 
qu'avoir des pointeurs entre noeuds de l'arbre, cette solution utilise une collection pour 
stocker les données (ici un vecteur) et les relations entre noeuds sont tenues grâce aux 
identifiants dans cette même collection (ici les indices du vecteur). Ainsi, le problème 
de durée de vie des valeurs et des multiples \textit{ownership} est résolu, car l'\textit{arena}
est la seule détentrice des données (donc des noeuds). Les noeuds restent accessibles depuis 
l'extérieur de l'\textit{arena} en gardant leurs identifiants dans l'\textit{arena}.
\bigbreak
\begin{code}
    \begin{minted}[bgcolor=mygray,breaklines,breaksymbol=,linenos,frame=single,stepnumber=1,tabsize=2]{rust}
pub struct Arena<T> {
    nodes: Vec<Node<T>>,
}

pub struct Node<T> {
    parent: Option<NodeId>,
    previous_sibling: Option<NodeId>,
    next_sibling: Option<NodeId>,
    first_child: Option<NodeId>,
    last_child: Option<NodeId>,
    /// The actual data which will be stored within the tree
    pub data: T,
}

pub struct NodeId {
    index: usize,
}
    \end{minted}
    \caption{Structure d'un noeud en Rust avec une \textit{arena} - \cite{ref49}}
    \label{rust_leipzig_arena}
\end{code}
\bigbreak
Le \textit{crate} petgraph utilisé pour stocker les données des fichiers, répertoires et tags d'une 
arborescence fonctionne de cette manière : il utilise deux vecteurs, l'un pour les noeuds, l'autre 
pour les arcs, pour stocker les données. Ce \textit{crate} répondait parfaitement aux besoins de 
la nouvelle architecture pour l'index des fichiers, répertoires et tags : il fournit une implémentation 
d'un graphe et permet l'accès aux noeud de l'extérieur du graphe.
Pour plus de détails sur cette sous-section, ces différents articles abordent cette problématique 
et donnent davantage de détails sur les possibles solutions \cite{ref2} \cite{ref26} \cite{ref49} 
\cite{ref46} \cite{ref47} \cite{ref48}.

%%%%%%%%%%%%%%%%%%%%%%%%%%%%%%%%%%%%%%%%%%%%%%%%%%%%%%%%%%%%%%%%%%%%%%%%%%%%%%%%%%%%%%%%%%%%%%%%%%%
%%%%%%%%%%%%%%%%%%%%%%%%%%%%%%%%%%%%%%%%%%%%%%%%%%%%%%%%%%%%%%%%%%%%%%%%%%%%%%%%%%%%%%%%%%%%%%%%%%%
\newpage
\subsection{Résultats et améliorations futures}
Les résultats des mesures de performances réalisés à la sous-section \ref{mesures_performances} 
montrent que le parcours initial du graphe est relativement rapide, tout du moins se déroule 
avec très peu de latence même pour un répertoire contenant plus de 15'000 répertoires et plus de 
100'000 fichiers (une moyenne d'exécution de deux secondes pour scanner toute l'arborescence). Il 
est raisonnable de dire que cette opération est parmi celles les plus lourdes pour une arborescence 
donnée, le programme peut donc être labélisé performant. Cependant, pour s'en assurer totalement, 
il faudrait réaliser des tests supplémentaires lors de plus longues utilisations.
\bigbreak
Bien que le cahier des charges ait été rempli, l'état du système au moment du rendu n'était pas parfait, 
voici une liste non exhaustive d'améliorations qui pourraient être apportées :
\begin{itemize}
    \item Intégration à un gestionnaire de fichiers Linux (Nautilus, Nemo, etc.) pour la manipulation
        des tags (ajout, suppression) et pour exécuter les requêtes. Une alternative serait de réaliser 
        une interface similaire sous forme d'application web par exemple.
    \item Transformation de Tag Engine en \textit{Daemon} (un processus détaché du shell parent et 
        qui n'imprime pas ses résultats sur la sortie standard). Il pourrait se lancer au démarrage 
        de l'\acrshort{os} et écrire ses résultats et indexes dans des fichiers.
    \item Possibilité d'ajouter des nouveaux chemins de répertoires à surveiller après le lancement 
        initial de Tag Engine. Actuellement, le programme se lance avec un seul chemin de répertoire
        à surveiller. Il faut pour cela modifier la gestion des threads et des données dans Tag 
        Engine. Le problème peut être partiellement résolu en donnant à Tag Engine un répertoire 
        suffisamment haut dans la hiérarchie pour surveiller un plus grand nombre de fichiers.
    \item Gestion des périphériques amovibles et de leurs \acrshort{fs}. \mintinline{bash}{inotify} ne 
        permet pas de surveiller le dossier \mintinline{text}{/media} de la même manière qu'un 
        autre répertoire. Il faut donc pouvoir détecter le branchement d'un périphérique de stockage 
        et ajouter une nouvelle surveillance sur le répertoire monté.
    \item Réaliser davantage de tests, plus globaux que des tests unitaires et de performances.
    \item Créer un cache des dernières requêtes logiques adressées au serveur, pour accélérer la 
        réponse. Le cache devrait être effacé en cas de modification de l'arbre ou de la \textit{hashmap}.
\end{itemize}
