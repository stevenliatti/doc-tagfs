Les deux objectifs principaux de ce projet de Bachelor étaient d'étudier et s'approprier le 
langage Rust et de concevoir un moteur de gestion de tags efficace et \textit{user friendly}.
Rust est un langage moderne, fiable et performant. Ses atouts sont aussi nombreux que les nouveaux 
concepts qu'il introduit par rapport à un langage comme C. Fort d'une communauté active et sérieuse, 
il y a l'espoir qu'il soit adopté par de plus en plus de développeurs pour de nombreux types d'applications.
Bien que l'étude de Rust ait pris une grande partie du temps alloué à ce travail, ce ne fut pas du 
temps perdu. Les contraintes imposées par Rust devraient être un standard bénéfique pour de nombreux 
langages. 
\bigbreak
D'autres sujets ont été étudiés, comme les méthodes d'indexation, les attributs étendus des fichiers, 
ou les systèmes de surveillance du système de fichiers. Les limites des attributs étendus ont été 
montrées (notamment sur l'incompatiblité avec certains systèmes de fichiers ou partages réseaux) 
et du système de notification \mintinline{text}{inotify}, choisi pour ce projet. Néanmoins, ces 
deux technologies, avec l'association de Rust, ont permis de surpasser sur certains points les 
applications existantes de gestion des tags. Le cahier des charges demandé 
a été rempli et les interrogations sur l'usage de Rust dans ce genre d'applications vérifiées. 
Grâce à ce système, l'utilisateur peut maintenant étiqueter ses fichiers personnels sans avoir 
peur de les perdre et peut les retrouver facilement et rapidement à l'aide de requêtes logiques 
simples. 
\bigbreak
Ce projet est le couronnement de mes études à hepia, il m'a fait découvrir un nouveau langage 
plein de potentiel, enseigné de bonnes pratiques de programmation et m'a fait progresser dans 
la démarche de conception et réalisation d'une application système. Tout le projet est disponible 
à cette adresse : \url{https://github.com/stevenliatti/tagfs}.
