%%%%%%%%%%%%%%%%%%%%%%%%%%%%%%%%%%%%%%%%%%%%%%%%%%%%%%%%%%%%%%%%%%%%%%%%%%%%%%%%%%%%%%%%%%%%%%%%%%%
%%%%%%%%%%%%%%%%%%%%%%%%%%%%%%%%%%%%%%%%%%%%%%%%%%%%%%%%%%%%%%%%%%%%%%%%%%%%%%%%%%%%%%%%%%%%%%%%%%%
\subsection{Motivations}
Avec l'augmentation de la puissance de calcul et la capacité de stockage de masse grandissante à un 
prix raisonnable, nos ordinateurs gèrent des quantités de fichiers très importantes, de l'ordre du 
millier ou du million de fichiers. Que ce soit des images, des documents ou de la musique, les 
réserves de stockage de fichiers semblent sans fin. Dès lors se pose la question de l'organisation 
de ces nombreux fichiers. Comment doit-on ordonner ses photos personnelles ? Par date, par lieu, 
par thème ? Ce sont trois bonnes réponses, mais malheureusement pour l'utilisateur, les \acrshort{os} 
d'aujourd'hui ne proposent qu'une seule manière native d'organiser ses fichiers : 
la classique hiérarchie de répertoires, sous-répertoires et fichiers, sous la forme d'une arborescence.
\bigbreak
Comment retrouver rapidement la photo de votre chat dormant sur votre balcon au début de sa vie 
dans une masse de plus de 10'000 images ? Comment classer son répertoire d'études, un répertoire 
pour les cours, un autre pour les travaux pratiques, ou pour chaque cours, deux sous-répertoires 
"théorie" et "pratique" ? Comment récupérer une chanson sans connaître son titre ni l'artiste mais 
en connaissant le genre ? Une solution à ces problèmes est de donner la possibilité à l'utilisateur 
d'apposer une ou plusieurs étiquettes, ou "tags", sur ses fichiers et de lui fournir une interface 
avec laquelle il pourra aisément retrouver ses fichiers. Il doit garder le contrôle sur ses fichiers 
et pouvoir les manipuler comme il l'a toujours fait. Les tags doivent être stockés avec les fichiers, 
pour qu'ils ne soient pas perdus en cas de grand changement dans le système. L'utilisation des 
attributs étendus, ou \textit{extended attributes} est la manière la plus naturelle de répondre à 
ce dernier besoin : les tags "voyagent" ainsi avec les fichiers. Nous verrons qu'il existe 
des applications résolvant en partie ce problème. Finalement, cette interface doit être performante et fiable. 
\bigbreak
Ces deux qualificatifs, performant et fiable, résument le langage de programmation Rust. Rust est 
un langage de programmation moderne fort d'une communauté grandissante et passionnée. Il est très 
performant, proche ou meilleur que C selon les situations. Il est fiable grâce à ses règles strictes 
sur l'utilisation de la mémoire et son compilateur très intelligent. C'est un langage totalement 
adapté à notre situation. À travers ce travail, le lecteur pourra se faire une illustration des 
possibilités offertes par Rust.
\bigbreak
Le but de ce travail est donc de concevoir et développer un moteur de gestion des tags répondant 
aux besoins cités précédemment et d'apprendre les notions de Rust nécessaires à sa réalisation.
%%%%%%%%%%%%%%%%%%%%%%%%%%%%%%%%%%%%%%%%%%%%%%%%%%%%%%%%%%%%%%%%%%%%%%%%%%%%%%%%%%%%%%%%%%%%%%%%%%%
%%%%%%%%%%%%%%%%%%%%%%%%%%%%%%%%%%%%%%%%%%%%%%%%%%%%%%%%%%%%%%%%%%%%%%%%%%%%%%%%%%%%%%%%%%%%%%%%%%%
\newpage
\subsection{Buts}
Avec plus de détails, les buts de ce projet sont les suivants :
\begin{itemize}
    \item Étudier et s'approprier le langage Rust pour la réalisation d'une application système sous Linux.
    \item Répertorier les applications existantes permettant d'étiqueter les fichiers.
    \item Étudier les \textit{extended attributes} lors des manipulation courantes sur les fichiers.
    \item Explorer les méthodes de surveillance du \acrshort{fs}.
    \item Analyser les moyens d'indexer une arborescence de fichiers.
    \item Concevoir et implémenter un système répondant aux motivations. Il devra être performant, 
        utilisable en temps réel et gérer de nombreux fichiers, répertoires et tags.
    \item Mesurer les performances de ce système.
    \item Réaliser une démonstration.
\end{itemize}
